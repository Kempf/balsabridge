\documentclass[12pt]{article}
	\usepackage{graphicx,amsmath,textcomp}
	\usepackage[round]{natbib}
	\usepackage[margin=3cm]{geometry}
	\linespread{1.3}
	\bibliographystyle{plainnat}
	\title{
		Group \#07 \\
		Bridge-00 \\
		18g, 1700N \\[1cm]
		\includegraphics[width=0.7\textwidth]{photo}
	}
	\author{
		Alex Miles \\ u5568175 \\ 16.7\%
		\and Arlene Mendoza \\ u5589650 \\ 16.7\%
		\and Itsuki Nishida \\ u5578430 \\ 16.7\%
		\and Paul Apelt \\ u5568225 \\ 16.7\%
		\and Stephen Lonergan \\ u5349877 \\ 16.7\%
		\and Thomas Hale \\ u5568225 \\ 16.7\%
	}
\begin{document}
	\maketitle
	\thispagestyle{empty}
	\setcounter{page}{0}
	\section{Design}
		% design assumptions and methods
		\subsection{Assumptions}
		During the design process, the following assumptions have been made \citep[p.~264]{tbook}:
		\begin{enumerate}
			\item All loadings are applied at the joint,
			\item Weight of the members neglected,
			\item Joints are smooth (friction-less) pins,
			\item Each member has no more than two joints.
		\end{enumerate}

		\subsection{Considerations}
The task was set with a design criteria to span a gap of 180~mm and a width of 70~mm. The entire bridge also the bridge should weigh less than 18~g. The goal was to be able to hold the largest weight possible.

A very important design consideration was how many 2D truss panels should be used. The idea was by having more panels the design would be distributing the weight and if the layers span the 70~mm the design would be more efficient with its wood use. This is because we no longer need pieces holding the trusses together that don't add significantly to the strength of the design. This would be a better use of the wood as all of our wood would be focused on holding a load applied from above. 

Starting with a 70~mm wide layer of Warren trusses spanning 180~mm ran into a few problems. The largest limitation on the design was the materials allowed this meant the more triangles in the Warren truss the lower the truss had to be. This would create more smaller triangles which would increase the material. Several designs were looked at but it was found that spanning the gap required using a Warren truss didn't work due to the limit of material. However by removing all the triangles we had the outside of the truss which was in the shape of a trapezium. 

Another factor was also making sure that the force was applied onto the joints and not onto one member therefore the top of the bridge where the loading is applied should be less than 70~mm. The top of the bridge runs into the supports on both sides of the bridge and this transfers the load directly into the ground on either side of the trapezium. This takes advantage of the surface reactions. Because of this the more vertical the side pieces the better as they will distribute more of the force onto the support reactions rather than the other sections of the bridge. This is under the assumption that the sides don’t break but if more force is downwards then some of the bracing stopping the bottom from slipping outwards wouldn't be needed and could instead could be moved to the sides.  

As these side pieces are taking the largest force it was decided to use the 5~mm pieces here. To stop the pieces from sliding out some of the 3~mm pieces on the bottom were used to hold the two ends of the design together and prevent slipping.The original design had half the pieces on the bottom as there were on the top. This bridge was tested rather imperfectly as the weight was skewed off centre and so broke earlier than expected. 

By moving two pieces of 5~mm to the sides of the top it was found that we could double the number of pieces along the bottom. Theoretically the top and bottom of the truss take the same force and this is shown later in the calculations. By moving the 5~mm pieces to the top the top was 100~N stronger than the bottom and so the wood was being used efficiently as this gap is relatively small. The side supports while stronger would actually fail before the top does but after the bottom does and therefore the wood is being used very efficiently. The theoretical predictions for this later in the report shows that this is indeed the case.

One thing also considered but decided against was gusset joints, it was decided that gusset joints would not be used for for two reasons. First of all was the material constraints, however this wasn't the biggest concern. The biggest concern was that gusset jointing due to how the joints are situated would be very similar to lamination, which was not allowed.

\subsection{Construction}
All the pieces were measured to the lengths in Table~\ref{list}, and they were then cut. Although the measuring was accurate, you lose some material whenever you cut. So to minimize loss of length, the pieces need to be cut in priority of one another. The 3~mm longest bottom pieces were cut first as these were extremely important to be at a full length. Total amount of material used is recorded in Table~\ref{materials}.
		\begin{table}[h!]
			\caption{List of member lengths.}
			\begin{center}
			\begin{tabular}{ | r | l | r | r | }
				\hline
				qty. & type & size & length \\ \hline
				7 & bottom & 3$\times$3~mm & 187~mm \\ \hline
				7 & top & 3$\times$3~mm & 70~mm \\ \hline
				2 & top & 5$\times$5~mm & 70~mm \\ \hline
				16 & sides & 5$\times$5~mm & 103.75~mm \\ \hline
			\end{tabular}
			\end{center}
			\label{list}
		\end{table}
		\begin{table}[h!]
			\caption{Materials used.}
			\begin{center}
			\begin{tabular}{ | r | r | r | }
				\hline
				size & given & used \\ \hline
				3$\times$3~mm & 1800~mm & 1799~mm \\ \hline
				5$\times$5~mm & 1800~mm & 1800~mm \\ \hline
			\end{tabular}
			\end{center}
			\label{materials}
		\end{table}

The leftover pieces were used to create the top of the bridge, these have the least priority as their length can be compromised if need be. Two 5 ~mm pieces are used on the top of the bridge. The 5~mm sides should be cut to the maximum length possible using the leftover material. Making them taller will distribute the more force into the support reactions. A small reduction in length would however not drastically reduce the maximum load. 

A template/scale diagram of the trapezium shape was then made to gain accuracy in the angles that needed to be attained (56\textdegree). Place the sides of the trapezium (5~mm) down with the top and bottom 3~mm pieces on top. Ignore the lone 5~mm top piece as this will be glued on last for ease of construction. Once enough glue (an amount when the two wood pieces are placed against each other that glue comes out from either side) was applied to the wood, the pieces were aligned to be at angles to each other and that the top and bottom pieces were in parallel. Weights were then placed to apply pressure to the joint. This would ensure the glue attains its maximum strength. The joints were then left to dry for each layer around 10 mins for another layer to be placed on top. The entire process was repeated until 8 layers were made.  The two 5~mm top pieces were then added to either side.


		Final bridge design can be seen in Figures \ref{dim} and \ref{proj}.
		\begin{figure}[h!]
			\centering
			\includegraphics[width=\textwidth]{dim}
			\caption{Dimensioned drawing.}
			\label{dim}
		\end{figure}
		\begin{figure}[h!]
			\centering
			\includegraphics[width=0.5\textwidth]{proj}
			\caption{3D-Projection.}
			\label{proj}
		\end{figure}
		% dimensioned drawing
		\subsection{Methods}
		% construction process 
		% materials Used
		% difficulties faced - measuring accurately, It ended up being 185~mm than 187~mm and the trapezium was also slightly lopsided by a 1~mm, angles are fucked!
		% weight issues - excess glue fail 
		% picture of the materials used
	\section{Analysis}
		Due to the unconventional design, to ease the calculations during the analysis, it was assumed that the load is equally distributed between eight trapezium-shaped trusses. Thus, a single trapezium truss was analyzed, and then extended to approximate the entire bridge. Internal forces, nodes and members are labelled as per Figure~\ref{trap}. As the truss is symmetrical, only two nodes needed to be analyzed.
		\begin{figure}[h!]
			\centering
			\includegraphics[width=0.7\textwidth]{trapanal}
			\caption{Trapezium truss.}
			\label{trap}
		\end{figure}
		For a load of 100N divided equally between nodes A and B, force equilibrium for nodes A and C are described in equation blocks \ref{eqn1} and \ref{eqn2} respectively.
		% insret calculations here
		\begin{subequations}
			\begin{align}
				F_s \sin 56&=50\mathrm{N}, \\
				F_s \cos 56&=F_t.
			\end{align} \label{eqn1}
		\end{subequations}
		\begin{subequations}
			\begin{align}
				F_s \sin 56&=F_r, \\ 
				F_s \cos 56&=F_b.
			\end{align} \label{eqn2}
		\end{subequations}

		The results of solving the above equations are presented in Table~\ref{loads}. Note that the internal force experienced by members is twice the given value, because it occurs at both ends.
		% table of member loads
		\begin{table}[h!]
			\caption{Member loads.}
			\begin{center}
			\begin{tabular}{ | r | l | }
				\hline
				70 3$\times$3 ~mm (top) & 33.7 N (c) \\ \hline
				103.75 5$\times$5 ~mm (side) & 60.3 N (c) \\ \hline
				187 3$\times$3 ~mm (bottom) & 33.7 N (t) \\ \hline
			\end{tabular}
			\end{center}
			\label{loads}
		\end{table}

		% assumptions
		% max load
		Maximum loads for each member were calculated using the values given in the Assignment sheet. Modulus of elasticity $E=3\mathrm{GN}/\mathrm{m}^2$, standard deviation $\sigma=+2.4/-2.1\mathrm{MN}/\mathrm{m}^2$. Tensile strength $\sigma_t=20\mathrm{GN}/\mathrm{m}^2$,  standard deviation $\sigma=+3.6/-3.4\mathrm{MN}/\mathrm{m}^2$. Compressive strength $\sigma_t=12\mathrm{GN}/\mathrm{m}^2$,  standard deviation $\sigma=+2.1/-2.8\mathrm{MN}/\mathrm{m}^2$. To calculate maximum load from strength values, equation \ref{eqs} was used. The results are presented in Table~\ref{maxloads}.
		\begin{equation}
			\sigma=\frac{P}{A}
			\label{eqs}
		\end{equation}
		\begin{table}[h!]
			\caption{Maximum member loads.}
			\begin{center}
			\begin{tabular}{ | r | l | l | }
				\hline
				& average & $-1\sigma$ \\ \hline
				3$\times$3 ~mm (c) & 108 N & 82.8 N \\ \hline
				3$\times$3 ~mm (t) & 180 N & 149.4 N \\ \hline
				5$\times$5 ~mm (c) & 300 N & 230 N \\ \hline
			\end{tabular}
			\end{center}
			\label{maxloads}
		\end{table}

		Buckling loads were calculated for members under compression (using equation~\ref{eqb}), and it found that the buckling loads of members under compression (see Table~\ref{buck}) did not exceed maximum load, therefore buckling was not critical.
		\begin{equation}
			P_b=\frac{\pi^2 E I}{(kL)^2}
			\label{eqb}
		\end{equation}
		\begin{table}[h!]
			\caption{Maximum buckling loads.}
			\begin{center}
			\begin{tabular}{ | r | l | l | }
				\hline
				& average & $-1\sigma$ \\ \hline
				3$\times$3 70 ~mm (c) & 163.1 N & 163.0 N \\ \hline
				5$\times$5 103.75 ~mm (c) & 573.1 N & 572.7 N \\ \hline
			\end{tabular}
			\end{center}
			\label{buck}
		\end{table}
		% reasons

		To predict the maximum load, the weakest section of the bridge had to be found. That was done by first su~mming up the maximum loads of all top, bottom and side members, (equation block~\ref{eqsa}), using values from Table~\ref{maxloads}, average column.
		\begin{subequations}
			\begin{align}
				P_{top}&=7\times108+2\times300
				&=1356\mathrm{N}\\
				P_{side}&=8\times300
				&=2400\mathrm{N}\\
				P_{bottom}&=7\times180
				&=1260\mathrm{N}
			\end{align}
			\label{eqsa}
		\end{subequations}

		Then the safety rations (defined in equation~\ref{eqr}) were calculated for each section (equation block~\ref{eqrl}), using values from Table~\ref{loads}.
		\begin{equation}
			R=\frac{P_{max}}{P}
			\label{eqr}
		\end{equation}
		\begin{subequations}
			\begin{align}
				R_{top}&=40.2\\
				R_{side}&=39.8\\
				R_{bottom}&=37.4
			\end{align}
			\label{eqrl}
		\end{subequations}

		The lowest safety ratio member will break first, thus bottom was determined to be the breaking point (snapping-type failure). The actual maximum load was determined in equation block~\ref{eqt}, where $P$ is the load used to calculate $R$. The same steps were carried out in equation block~\ref{eqs}, but the strength was assumed to be $1\sigma$ below average.
		\begin{subequations}
			\begin{align}
				F&=\frac{R P}{2} \\
				&=1869.4 \mathrm{N}
			\end{align}
			\label{eqt}
		\end{subequations}
		\begin{subequations}
			\begin{align}
				F&=\frac{R_{\sigma} P_{\sigma}}{2} \\
				&=1551.6 \mathrm{N}
			\end{align}
			\label{eqs}
		\end{subequations}

		The average of the two $1710.5\mathrm{N}$, was chosen as predicted maximum load.
	\section{Results}
The predicted strength of the bridge was 1716~N while the maximum force that it sustained during the test was 875~N. The most significant contributor to this imprecise prediction was that the bridge was too short to comfortably span the 18~cm gap. Furthermore, gluing techniques were not researched and as a result the joints were not as strong as they otherwise may have been.

Upon a load being applied to the top of the bridge, it became evident that the rear left corner did not have the length to fully span the gap. The rear 4 trapeziums immediately started to slip down into the gap below the block. This was ultimately the cause of the actual load being significantly below the predicted load. The four supported trapeziums at the front of the structure still held, and they performed as they were expected to. However, no member of the bridge actually failed. Other minor factors played parts in this, like the bridge being slightly higher at the top in the rear, and the overall strength of the glued joints. In an attempt to analyse why no member broke, the full sequence of events is laid out.

To prevent the rear left section of the bridge from instantly separating and falling down under the load, the joints needed to supply an upward force, equivalent to the downward force on the section. This resulted in a shear force on the 5th joint on the upper left hand side, a shear force on the 4th joint on the lower left hand side, and moments about the corresponding joints on the left hand side. 

		\begin{figure}[h!]
			\centering
			\includegraphics[width=0.5\textwidth]{failtop}
			\caption{Failure points on the top of the bridge.}
			\label{failtop}
		\end{figure}

The balsa cement itself has a reasonably high degree of elasticity under both torsional and shear forces, with the ability to be displaced several millimetres prior to failure. \citep{glue}.

The moment about the two right-side joints was also counteracted by the glue. As was shown in our tests, the maximum torsional force (light fibre-tear failure) is roughly one third the maximum shear force (cohesive failure). Double gluing may possibly have spread this torsional force through a larger volume of cement, allowing it to withstand a larger force prior to failure. 

The joints on the left hand side however, were more susceptible than the right hand side. The joints on the left had separated by about 2~mm prior to failure. Due to the much higher shear strength this was unexpected, however, the cements high elasticity was what enabled this. The span of the bride meant that the right joints had only rotated between 0\textdegree and 2\textdegree (loose estimate considering the 2~mm drop on the left side), while during testing they were able to maintain rotation of up to 45\textdegree prior to failure, which was when the max force was reached. There was also a small shear force on each joint on the right, but the glue was able to absorb this at this time, as it was significantly less than the shear on the left joints.

After the glue had reached a critical length on the left hand joints, it sheared, leaving little to no support on this side for the first 4 trapeziums, which was shown by them sinking significantly into the gap.

The only support the section had at this time on was from the two joints on the right hand side, which were now experiencing a large moment, combined with a shear. 

Through testing, the maximum moment for a heavily glued joint was about $28$ -- $30\mathrm{Nm/cm^2}$ of cement. The glue was estimated to have been applied to about $1\mathrm{cm^2}$, the maximum moment in a joint would be about $28$ -- $30\mathrm{Nm}$ and the maximum shear about $65$ -- $80\mathrm{N}$. 

The trapeziums appear to flatten out as the top moves to the left, so the load is still distributed at this point. However, due to the slant of the bridge, the rear trapeziums had a much higher load than the front ones, causing a large shear force to build up in the upper and lower right joints, going from the rear to the supported section as they failed one by one. This suggests that the joints did not fail due to the moment. ($30 / 0.035 = 8570\mathrm{N}$) which would be the required resultant distributed force half way along the top of the bridge. 

The force exerted on each trapezium would need to exceed a minimum of 130~N, to shear the two joints. It appears that the top joint would fail, allowing the trapezium to shift in shape farther to the left. This converted the shear in the lower joint to a torsional force, which was absorbed by the glue. 

The joints of concern all failed within this range, the distributed load gradually evened out as the top flattened, but the load was estimated to be under 200~N by the time the last joint failed. 
Consequently the rear 4 trapeziums were supporting almost none of the load. This was illustrated when the loading was removed and the bridge does not return to its original position.

This left the remaining force to be supported by the functioning 4 trapeziums, which gradually took the load as was predicted, with the bottom spreading slightly as the tension in the lower members was increased. Having only one half of the bridge remaining, the maximum load dropped to half the predicted value ($1716 / 2 = 850\mathrm{N}$). Had the bridge not been too short a failure should have occurred around this region. It was predicted that a lower member would break, however, none did due to the joint failures. This leaves that the final four trapeziums crossed the max threshold for the elasticity of the cement in the joints, and began to slip, (due to the large volume of glue, it has strong elastic deformation. This value was roughly equivalent to the breaking force of the lower members (10 upper moments and 7 lower shear forces/moments; this is the number of $\mathrm{cm^2}$ in joints, $3.5 \mathrm{cm^2}$ on each side) This gives a value of ($875\cos 56 = 489\mathrm{N}$). The value for failure of 7 shear joints is $65$ -- $80\mathrm{N}$ each, giving a value between $455$ and $560\mathrm{N}$. Due to the small torsional force in the joints, on top of the shear, the value is in the lower range of the estimate. A deviation in the actual surface area of the glue creates some room for error in this. Though, this range matches the tested value.  At this point, a joint slipped, registering a decreased load, stopping the machine and preventing a structural failure.
\clearpage
\bibliography{bib}
\clearpage
\section*{Appendix}
%glue stuff here
\end{document}